\section{Microcontroller Board}
\begin{itemize}[noitemsep]
  \item Adjust the board to accommodate the Teensy instead of STM32-H405. This would get rid of a big part of wires, make the electronics box cleaner and more presentable, reduce the amount of space needed, and also possibly reducing the risk of wires coming loose.
  \item Four wires are soldered to the PCB instead of using a header. Adjusting the PCB for this would make the PCB look more presentable.
\end{itemize}

\section{Power Board}
\begin{itemize}[noitemsep]
  \item Some cables running to/from this board might need extending and replacing the connectors.
  \item The cables carrying power to the fork motor are the ones requiring some care the most.
\end{itemize}

\section{Protoboard}
\begin{itemize}[noitemsep]
  \item Some of the wires going to the protoboard are soldered straight to it without any connectors in-between. This means that taking the electronics box apart is a hassle since de-soldering is needed. Installing connectors would improve this.
  \item A custom PCB might make disassembly even easier.
  \item The sensors were not tested out nor calibrated during the MPC project as they were not needed. Before making use of the sensors -- calibrate the sensors.
\end{itemize}

\section{Bicycle}
\begin{itemize}[noitemsep]
  \item Adjust front brakes to not rub, if the bicycle is to be used outside on the road.\footnote{The brakes should be disabled while riding on the treadmill.}
  \item Weather-proof the motors.
  \item Fix the kick-stand. Currently it is not stable.
  \item Slick tires are not very suitable for riding outside on the road.
\end{itemize}
